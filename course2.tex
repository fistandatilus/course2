% !TeX spellcheck = ru_RU
\documentclass[a4paper,14pt]{extarticle}
\usepackage[T2A]{fontenc}
\usepackage[utf8]{inputenc}
\usepackage[english,russian]{babel}
\usepackage{amsmath}
\usepackage{amssymb}
\usepackage{amsthm}
\usepackage{listings}

\usepackage{xcolor}

\usepackage{indentfirst}
\usepackage[left = 20mm, right = 10mm, top = 20mm, bottom = 20mm]{geometry}
\usepackage[indent=1.25cm]{parskip}
\renewcommand{\baselinestretch}{1.5}

\newtheorem{theorem}{Теорема}

\lstset{label=Листинг, tabsize = 2, basicstyle=\ttfamily, captionpos=b, escapeinside={№}{/№}}

\author{Борисенков Никита}



\begin{document}

\begin{titlepage}
    \newpage
    \begin{center}
        Московский государственный университет имени M.В. Ломоносова
        Механико-математический факультет\\
        Кафедра вычислительной математики \\
    \end{center}

    \vspace{8em}

    \begin{center}
        \Large Курсовая работа \\
    \end{center}

    \vspace{2em}

    \begin{center}
        \textsc{\textbf{}}
    \end{center}

    \vspace{20em}



    \newbox{\lbox}
    \newlength{\maxl}
    \setlength{\maxl}{\wd\lbox}
    \hfill\parbox{13cm}{
        \hspace*{5cm}\hspace*{-5cm}Студент: \qquad\qquad\hbox to \maxl{Борисенков Никита Николаевич\hfill}\\
        \hspace*{5cm}\hspace*{-5cm}Преподаватель: \hbox to \maxl{ С.н.с Кривчиков Максим Александрович}\\
        \\
        \hspace*{5cm}\hspace*{-5cm}Группа:\qquad\qquad $\;\:$ \hbox to\maxl{410}\\
    }


    \vspace{\fill}

    \begin{center}
        Москва \\2025
    \end{center}

\end{titlepage}
\newpage

\section*{Введение}

\section{Результаты известных исследований}

Задача выделения регистров является важной частью процесса компиляции программ. Самым известным способом решения этой задачи пожалуй является сведение её к раскраске графа пересечения времён жизни переменных. Эта идея была предложена в 1981 году в \cite{chaitin_register_1981}. Для произвольных графов задача раскраски является NP-сложной и в этой же статье показано, что для любого графа можно написать программу, задача выделения регистров которой будет сводиться к раскраске этого графа.

В статье \cite{hans_l_bodlaender_linear-time_1997} показано, что если число регистров фиксировано и не производится вынос переменных на стек, то для <<структурированных>> программ существует линейный по количеству вершин и рёбер в графе алгоритм выделения регистров. Структурированность в данной статье означает отсутствие \texttt{goto}-переходов. Из структурированности следует ограниченная древесная ширина графов, а для таких графов существует линейный алгоритм определения возможности и нахождения раскраски. Коэффициент в линейной зависимости получается более, чем линейно зависящим {\color(red) это нормальная фраза?} от числа регистров.

Другой известный способ выделения регистров -- это линейный скан {\color{red} (это правильный перевод?)}, предложенный в статье \cite{poletto_linear_1999}. Этот алгоритм не использует графовое представление пересечений интервалов жизни переменный, а работает с ними на прямую. Сложность этого алгоритма линейна по количеству переменных, что является значительным плюсом по сравнению с раскраской графа, и позволяет его использовать в случаях, когда время компиляции важно (например, JIT-компиляция). Минусом этого алгоритма является худшее качество кода, по сравнению с раскраской графа.

\section{Постановка задачи}

\section{Решение}

\section{Заключение}

\newpage

\bibliographystyle{ugost2008}
\bibliography{fclang_register_allocation}


\end{document}
