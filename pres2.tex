
\documentclass[10pt,a4paper]{beamer}
\usepackage[T2A]{fontenc}
\usepackage[utf8]{inputenc}
\usepackage[english,russian]{babel}
\usepackage{amsmath}
\usepackage{amsfonts}
\usepackage{amssymb}
\usepackage{xcolor}
\usepackage{listings}

\lstset{tabsize = 2, basicstyle=\ttfamily, escapeinside={№}{/№}}

\date{}

\begin{document}
\maketitle

\begin{frame}
\frametitle{Frame template}

\end{frame}

\begin{frame}
\frametitle{Введение}

То, что есть такой предложенный язык.
То, что я делаю систему выделения регистров.
То, что я делал макет, потому что до меня ничего не было написано.

\end{frame}

\begin{frame}
\frametitle{Актуальность}
чут-чут со ссылкой, зачем нужен язык
дальше то, что в нём есть переменные и без выделения регистров не сделать переменные никак.

\end{frame}

\begin{frame}
\frametitle{Известные исследования}
Сначала то, что очевидно, что эта задача так или иначе решалась
\end{frame}

\begin{frame}
\frametitle{Известные исследования. Раскраски графов}
Про то что это самое раннее, что я нашёл.
Основной принцип.
NP-сложность.
\end{frame}

\begin{frame}
\frametitle{Известные исследования. Раскраски графов}
Более простые алгоритмы для ограниченных множеств графов.
Хордальные графы.
Графы ограниченной древесной ширины.
Какие ограничения на код при этом?
Факт о том, что в произвольной программе с goto возможен любой граф.
\end{frame}

\begin{frame}
\frametitle{Известные исследования. Линейные сканирования}
Замечание про оптимальность
Про ELS и почему не подходит.
\end{frame}

\begin{frame}
\frametitle{Известные исследования. LLVM}

\end{frame}

\begin{frame}
\frametitle{Постановка задачи}
Нужно написать алгоритм выделения регистров с учётом особенностей FCLang.
Нужно написать макет языка FCLang, чтобы было над чем работать.

Можно сюда вставить формальную задачу о выделении регистров.
\end{frame}

\begin{frame}
\frametitle{Выбранный алгоритм}

Это линейное сканирование.
К нему прикручены условия, накладываемые явным использованием регистров.
Что я вообще видел про эти условия.

\end{frame}

\begin{frame}
\frametitle{Описание работы условий}
\end{frame}

\begin{frame}
\frametitle{Описание работы условий}
проблемы с дополнительным несрабатыванием.
история с перманентом и ещё одним алгоритмом на графах.
картинки к ним
\end{frame}

\begin{frame}
\frametitle{Макет яыка. Описание}
\end{frame}

\begin{frame}
\frametitle{Макет языка. Пример}
Возможно стоит ещё и сравнение с си, но боюсь не успею.
там нужно память сделать
\end{frame}

\begin{frame}
\frametitle{Как это можно развить}
Добавить SMID регистры.
Нынешняя реализация не позволят использовать части регистра.
С векторными регистрами это необходимо.

Можно строго формулировать требоваия к языку.
Можно реализовывать больше его частей.
\end{frame}

\begin{frame}
\frametitle{Заключение}
В этой работе был реализован алгоритм выделения регистров.
Был создан макет языка, который показывает работоспособность этого алгоритма и языка.
\end{frame}
\bibliographystyle{apalike}
\bibliography{}
\end{document}
